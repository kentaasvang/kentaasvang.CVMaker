%%%%%%%%%%%%%%%%%%%%%%%%%%%%%%%%%%%%%%%%%
% Twenty Seconds Resume/CV
% LaTeX Template
% Version 1.1 (8/1/17)
%
% This template has been downloaded from:
% http://www.LaTeXTemplates.com
%
% Original author:
% Carmine Spagnuolo (cspagnuolo@unisa.it) with major modifications by 
% Vel (vel@LaTeXTemplates.com)
%
% License:
% The MIT License (see included LICENSE file)
%
%%%%%%%%%%%%%%%%%%%%%%%%%%%%%%%%%%%%%%%%%

%----------------------------------------------------------------------------------------
%	PACKAGES AND OTHER DOCUMENT CONFIGURATIONS
%----------------------------------------------------------------------------------------

\documentclass[letterpaper]{twentysecondcv} % a4paper for A4
\usepackage[utf8]{inputenc}
\usepackage{graphicx}
%----------------------------------------------------------------------------------------
%	 PERSONAL INFORMATION
%----------------------------------------------------------------------------------------

% If you don't need one or more of the below, just remove the content leaving the command, e.g. \cvnumberphone{}

% \profilepic{$profilepicture} % Profile picture
\profilepic{} % Profile picture

\cvname{} % Your name
\cvjobtitle{} % Job title/career

\cvdate{} % Date of birth
\cvaddress{} % Short address/location, use \newline if more than 1 line is required
\cvnumberphone{} % Phone number
\cvmail{} % Email address
\cvsite{}
\cvsiteb{} % Personal website

%----------------------------------------------------------------------------------------

\begin{document}

%----------------------------------------------------------------------------------------
%	 ABOUT ME
%----------------------------------------------------------------------------------------

\aboutme{} % To have no About Me section, just remove all the text and leave \aboutme{}

%----------------------------------------------------------------------------------------
%	 SKILLS
%----------------------------------------------------------------------------------------
% Skill bar section, each skill must have a value between 0 an 6 (float)

\skills{}

% \skills{
%     {C\#.Net Core/3.5}, 
%     {PHP/3.0}
% }
%------------------------------------------------

% Skill text section, each skill must have a value between 0 an 6
%\skillstext{}

%----------------------------------------------------------------------------------------

\makeprofile{} % Print the sidebar

%----------------------------------------------------------------------------------------
%	 EDUCATION
%----------------------------------------------------------------------------------------



%----------------------------------------------------------------------------------------
%	 KURS OG SERTIFIKATER
%----------------------------------------------------------------------------------------

%----------------------------------------------------------------------------------------
%	 EXPERIENCE
%----------------------------------------------------------------------------------------



% \section{Erfaring}
% 
% \begin{twenty} % Environment for a list with descriptions
%     \twentyitem{2020-}{Lynx Publishing}{Fullstack Developer and DevOps}{Sole developer and DevOps engineer at Lynx Publishing, a content management system for newspapers. }
%     \twentyitem{2019-2020}{Optimeering}{Backend Developer}{Fullstack developer for an AI startup within the energy business. }
%     \twentyitem{2017-}{Skrypt Åsvang}{Independent Programmer}{app development for web and mobile. IT-consultancy, System development and maintenance.}
%     \twentyitem{2009-2018}{Different companies}{Process technician, operator}{In this time period I work for several different companies; Norwegian Crystals, IKM testing, Cannseal, Veidekke, Norsea, Maersk and REC. I gained a lot of domain knowledge from the industry, especially the oil and solar sector.}
% 	%\twentyitem{<dates>}{<title>}{<location>}{<description>}
% \end{twenty}





% \section{Utdanning}
% 
% \begin{twenty} % Environment for a list with descriptions
% %     \twentyitem{2016-2019}{Noroff's University}{Kristiansand/Nettbasert}{Bachelor of Applied Data Science}
% %     \twentyitem{2013-2016}{Nortrain's Technical college}{Stavanger}{Petroleum technology and production}
% %     \twentyitem{2012}{REC Mono}{Glomfjord}{certificate of apprenticeship as Process technician}
% %     \twentyitem{2012}{Nortrain's offshorekurs}{Stavanger}{Drilling technology}
% %     \twentyitem{2007-2009}{Glomfjord videregående}{Glomfjord}{Chemistry and industrial processes. Technical and industrial production}
% 	%\twentyitem{<dates>}{<title>}{<location>}{<description>}
% \end{twenty}

% \section{Kurs og sertifikater}
% this could be publications
%\begin{twentyshort} % Environment for a short list with no descriptions
%    \twentyitemshort{2009}{Fagbrev - produksjoneteknikk}
%	\twentyitemshort{2011}{Trucksertifikat (T1, T1, T4)}
%	\twentyitemshort{2012}{SRC/VHF sertifikat}
%	\twentyitemshort{2012}{Riggerbevis (Løfteredskap, fallsikring, søylesvingkran, taljer, rigging)}
%\end{twentyshort}
%----------------------------------------------------------------------------------------
%	 SECOND PAGE EXAMPLE
%----------------------------------------------------------------------------------------

%\newpage % Start a new page

% \makeprofile % Print the sidebar

%\section{Other information}

%\subsection{Review}

%Alice approaches Wonderland as an anthropologist, but maintains a strong sense of noblesse oblige that comes with her class status. She has confidence in her social position, education, and the Victorian virtue of good manners. Alice has a feeling of entitlement, particularly when comparing herself to Mabel, whom she declares has a ``poky little house," and no toys. Additionally, she flaunts her limited information base with anyone who will listen and becomes increasingly obsessed with the importance of good manners as she deals with the rude creatures of Wonderland. Alice maintains a superior attitude and behaves with solicitous indulgence toward those she believes are less privileged.

%\section{Other information}

%\subsection{Review}

%Alice approaches Wonderland as an anthropologist, but maintains a strong sense of noblesse oblige that comes with her class status. She has confidence in her social position, education, and the Victorian virtue of good manners. Alice has a feeling of entitlement, particularly when comparing herself to Mabel, whom she declares has a ``poky little house," and no toys. Additionally, she flaunts her limited information base with anyone who will listen and becomes increasingly obsessed with the importance of good manners as she deals with the rude creatures of Wonderland. Alice maintains a superior attitude and behaves with solicitous indulgence toward those she believes are less privileged.

%----------------------------------------------------------------------------------------

\end{document} 

